\documentclass[10pt]{article}
\usepackage{acl2014}
\usepackage[utf8]{inputenc}
\usepackage{times}
\usepackage{url}
\usepackage{amsmath}
\usepackage[natbib=true,backend=bibtex,style=authoryear,language=german]{biblatex}
\usepackage{lipsum}
\usepackage{latexsym}
\usepackage[normalem]{ulem}
\useunder{\uline}{\ul}{}
\usepackage{caption} 
\usepackage{lastpage}
\pagestyle{plain} 
\usepackage{graphicx}
\captionsetup[table]{skip=10pt}
\title{Intelligent Road Control}

\addbibresource{references.bib}
\AtBeginBibliography{\small}

\author{Thomas van den Broek, Vincent Van Driel, Zsolt Harsànyi, Tonio Weidler\\
	Department of Data Science \& Knowledge Engineering\\
	Maastricht, The Netherlands\\
	\tt \small vincevandriel@gmail.com, thomasbuo@gmail.com,\\
	\tt \small zsharsany@gmail.com, uni@tonioweidler.de
}
  
\begin{document}

\maketitle

\section{Introduction}
An intelligent traffic control system adjusts traffic in order to assure all people reach their destinations in the most optimal time and distance. These systems are important for the daily workings of major cities by alleviating traffic congestion and identifying problem areas.  It is important to constantly keep these systems updated to assure optimal performance and safety of the general public. With advancements in technology and artificial intelligence, new and more sophisticated strategies for traffic control become possible.

\subsection{Goal}
In this work, we aim to evaluate different such "intelligent" traffic control strategies and compare their effect. Thereby, we hope to identify the best working solutions to traffic jams and other occurrences (such as closing of roads/tunnels) in cities and to show their workings. We attempt to to recreate a real life city's traffic flow in order to see how these systems would work in a realistic setting. Our simulation aims to model the dynamics of a realistic environment as close as possible within the scope of the project, so that the comparison of strategies is meaningful and provides valuable data regarding the potential application of the strategies.

\subsection{Approach}
Hence, in order to test the effect of different intelligent traffic control strategies, an appropriate simulation is required. For this purpose, we created a simulation environment which allows the incorporation of different such strategies into a dynamic traffic model. Maps are realised as undirected graphs in which vertices represent intersections and roads appear as edges between those. The simulation is \textit{microscopic}. That is, instead of globally controlling traffic (\textit{macroscopic}), the atomic parts of the simulation are locally controlled cars \citep[see also][]{krajzewicz2002sumo}. Driving behaviour is modelled using the time- and space-continuous \textit{Intelligent Driver Model} (IDM) \citep{treiber2000congested}. 

\vspace{20pt}

In the following sections we further describe our approach and the results of our experiments. We begin by reviewing the literature. Based on that, we will elaborate on the implementation of the simulation environment (section \ref{sec:envi}), followed by a description of different control strategies (section \ref{sec:strategies}). Following, we describe the experiments we conducted for those strategies and our evaluation methodology (section \ref{sec:experiments}). In section \ref{sec:results} we will report the results of these experiments.  We conclude this work by discussing our results and proposing future directions.

\section{Related Work}
\label{sec:related-work}
There has been extensive work on the simulation of traffic flow as well as the development of traffic control strategies. Following up on different approaches on modelling car following behaviour \citep[e.g.][]{gipps1981behavioural}, \citet{treiber2000congested} developed the influential Intelligent Driver Model for the simulation of urban traffic. Similarly to the former, it creates a collision free environment where cars mind the a spacial and time-wise gap to the leading vehicle. The SUMO package \citep{krajzewicz2002sumo, behrisch2011sumo} utilizes the model by \citep{gipps1981behavioural} in an extended version \citep{krauss1998microscopic} in a complex simulation software. In contrast to their work, we use the IDM. Hence, our simulation is (quasi) time-continuous, rather than time-discrete.
	
\section{Simulation Environment}
\label{sec:envi}
In the following subsections we describe the implementation and design choices of the simulation environment in which we evaluated the effect of different traffic control strategies.

\subsection{Graphical User Interface}
The Graphical User Interface (GUI) is designed for simplicity. The left panel is used to show and create interactive maps. These maps can be interactively created by clicking and dragging on the panel, which creates intersections and the connecting roads. If the right mouse button is clicked the creating of the road will be cancelled. The roads currently have two sides, one for each direction. In the next phase there will be more lanes available. In between the two lanes there is a coloured bar, representing the traffic light. On the right side of the screen there is a button panel.

From top to bottom the buttons are:
\begin{itemize}
	\setlength\itemsep{0.2em}
	\item \textit{Clear}: clears the panel and triggers a model reset.
	\item \textit{Reset Position}: resets the graph to its original location and size should this have been altered.
	\item \textit{Plus and Minus}: allow the user to zoom in and out; arrow keys allow the user to move the graph.
	\item \textit{Add car}: randomly places cars on the map.
	\item \textit{Save and Load}: save and load road maps. 
	\item \textit{Start and Pause}: start and pause the simulation.
	\item \textit{Help}: provides explanation about the interface
\end{itemize}

During the interactive creation of a map, certain checks are run. Every intersection is restricted to not have more than four roads connected. If during two roads would cross each other, than at this crossing a new intersection is created.
	
\subsection{Traffic Flow Simulation}

As mentioned beforehand, we apply the IDM \citep{treiber2000congested} for the simulation of car dynamics. It models traffic flow time- and space-continuous as a combination of \textit{free-road} and \textit{interaction} behaviour. The \textit{free-road term} is governed by a car's intention to reach its desired speed. The acceleration for this behaviour is calculated \citep{treiber2000congested} as

\begin{equation}
	\label{eq:free-road-term}
	\dot{v}_a^{free}(t) = a ( 1 - ( \frac{v_a}{v_0} )^\delta),
\end{equation}

where $a$ refers to a cars maximum acceleration, $v_a$ is its current velocity and $v_0$ the desired velocity. When a car approaches a leading vehicle, it is supposed to slow down in order to avoid collision. This behaviour is modelled by an \textit{interaction term} which incorporates the distance to the leading vehicle and its speed \citep{treiber2000congested}. 

\begin{equation}
	\label{eq:int-term}
	\dot{v}_a^{int}(t) = - a ( \frac{s_0 + v_a T}{s_a} + \frac{v_a \Delta v_a}{2 \sqrt{ab} s_a} )^2
\end{equation}

In the equation above, $s_0$ and $T$ restrict the cars minimum distance in space and time respectively. The interaction term will hence attenuate the free road term when approaching other cars, given by the complete equation for acceleration $\dot{v}(t)$

\begin{equation}
	\dot{v}(t) = \dot{v}_a^{int}(t) + \dot{v}_a^{int}(t)
\end{equation}

In order to simulate a time-continuous model, we need to numerically approximate the integration of the differential equations in \ref{eq:free-road-term} and \ref{eq:int-term}. For that, we choose a small time step $\Delta t$ and repetitively update the velocity as $v(t + \Delta t) = v(t) + \dot{v}(t)$.

\subsection{Traffic Rule Compliance}
Due to the large scale maps used in this work, drivers not only need to behave according to their own desires and other traffic participants. They additionally need to comply to a set of traffic rules as given by speed limits or traffic lights. Following, we briefly discuss the approaches we use in our simulation.

\paragraph{Approaching Traffic Lights}

\paragraph{Speed Limits} The IDM's free road term incorporates a driver's desired velocity as his intended maximum speed. We use an additional parameter, a driver's favoured velocity $v_{fav}$, and determine the desired velocity by taking the minimum of $v_{fav}$ and the roads speed limit.


\subsubsection{Lane Changing}

\subsection{Arrival Times in a continuous Simulation}
Rather than predefining arrival times for a fixed simulation duration, we dynamically generate inter-arrival times (IAT) during runtime. These IATs are drawn from a predefined distribution independently for each road in the map. Therefore the amount of traffic on a map grows automatically with its size.

\section{Traffic Control Strategies}
\label{sec:strategies}

\subsubsection{Benchmark Strategies}

\subsubsection{Advanced Strategies}

\section{Methodology \& Experiments}
\label{sec:experiments}

\section{Results}
\label{sec:results}
	
\section{Conclusion}
\label{sec:conclusion}

{\tiny\printbibliography}

\end{document}